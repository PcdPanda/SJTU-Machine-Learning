\documentclass{article}
\usepackage{mathrsfs}
\usepackage{amsmath}
\usepackage{amsthm}
\usepackage{amssymb}
\usepackage{graphicx}
\usepackage{color}
%\include{macros}
%\usepackage{floatflt}
%\usepackage{graphics}
%\usepackage{epsfig}
\usepackage{float}%稳定图片位置
\usepackage{graphicx}%画图
\newcommand{\reals}{{\mathbb{R}}}
\newcommand{\dom}{{\bf{dom}}}
\newcommand{\symm}{{\bf{S}}}
\newcommand{\Tr}{{\bf{tr}}}

\theoremstyle{definition}
\newtheorem{theorem}{Theorem}[section]
\newtheorem{lemma}[theorem]{Lemma}
\newtheorem{proposition}[theorem]{Proposition}
\newtheorem{corollary}[theorem]{Corollary}

\theoremstyle{definition}
\newtheorem*{defition}{Definition}
\newtheorem*{example}{Example}

\theoremstyle{remark}
\newtheorem*{remark}{Remark}
\newtheorem*{note}{Note}
\newtheorem*{exercise}{Exercise}

\setlength{\oddsidemargin}{-0.25 in}
\setlength{\evensidemargin}{-0.25 in} \setlength{\topmargin}{-0.25
in} \setlength{\textwidth}{7 in} \setlength{\textheight}{8.5 in}
\setlength{\headsep}{0.25 in} \setlength{\parindent}{0 in}
\setlength{\parskip}{0.1 in}

\newcommand{\homework}[4]{
\pagestyle{myheadings} \thispagestyle{plain}
\newpage
\setcounter{page}{1} \setcounter{section}{#4} \noindent
\begin{center}
\framebox{ \vbox{\vspace{2mm} \hbox to 6.28in { {\bf
VE485,~Optimization~in~Machine~Learning (Summer 2020) \hfill Homework: #1} }
\vspace{6mm} \hbox to 6.28in { {\Large \hfill #1 \hfill} }
\vspace{6mm} \hbox to 6.28in { {\it Lecturer: #2 \hfill} }
\vspace{2mm} \hbox to 6.28in { {\it Student: #3 \hfill} }
\vspace{2mm} } }
\end{center}
\markboth{#1}{#1} \vspace*{4mm} }


\begin{document}

\homework{3. Convex Optimization}{Xiaolin Huang \hspace{5mm} {\tt
xiaolinhuang@sjtu.edu.cn}}{Chongdan Pan
\hspace{5mm} {\tt panddddda@sjtu.edu.cn } }{9}

%%%%%%%%%%%%%%%%%%%%%%%%%%%%%%%%%%%%%%%%%%%%%%%%%%%%%%%%%%%%%%%%%%%%
% Section 2.  Problem
%%%%%%%%%%%%%%%%%%%%%%%%%%%%%%%%%%%%%%%%%%%%%%%%%%%%%%%%%%%%%%%%%%%%

\section*{Problem 1} \label{ex-midpoint-cvx}
Consider the following \emph{Square LP},
\[
\begin{array}{ll}
\mbox{minimize}   & c^T x \\
\mbox{subject to} & Ax \preceq b,
\end{array}
\]
with $A$ square and nonsingular.
Show that the optimal value is given by
\[
p^\star = \left\{ \begin{array}{ll}
 c^TA^{-1}b & A^{-T}c \preceq 0 \\
 -\infty & \mbox{�������} \end{array} \right.
\]

{\bf{Answer:
\\$p^\star=c^TA^{-1}(Ax)=(A^{-T}c)^T(Ax)$
\\When $A^{-T}c\succ0$, $p^\star=-\infty$ when $Ax=-\infty$
\\When $A^{-T}c\preceq0$, $p^\star=c^TA^{-1}b$ when $Ax=b$
}}


\section*{Problem 2}\label{e-lin-frac-image}%
\emph{Relaxation of Boolean LP.}
In a \emph{Boolean linear program}, the variable $x$ is constrained
to have components equal to zero or one:
\begin{equation} \label{e-bool-lp}
\begin{array}{ll}
\mbox{minimize} & c^T x \\
\mbox{subject to} & Ax \preceq b \\
& x_i \in \{0,1 \}, \quad i=1,\ldots,n.
\end{array}
\end{equation}
In general, such problems are very difficult to solve, even
though the feasible set is finite (containing at most $2^n$ points).

In a general method called \emph{relaxation}, the constraint that
$x_i$ be zero or one is replaced with the
linear inequalities $0\leqslant x_i \leqslant 1$:
\begin{equation} \label{e-bool-lp-relax}
\begin{array}{ll}
\mbox{minimize} & c^T x \\
\mbox{subject to} & Ax \preceq b\\
& 0 \leqslant x_i \leqslant 1, \quad i=1,\ldots,n.
\end{array}
\end{equation}
We refer to this problem as the \emph{LP relaxation} of the
Boolean LP~(\ref{e-bool-lp}).  The LP relaxation is far easier
to solve than the original Boolean LP.

\begin{enumerate}
\item Show that the optimal value of the LP relaxation~(\ref{e-bool-lp-relax}) is a lower bound on the optimal value of the Boolean LP~(\ref{e-bool-lp}).  What can you say about the Boolean LP if the LP relaxation is infeasible?
    
\item It sometimes happens that the LP relaxation has a solution with $x_i \in \{ 0,1\}$.  What can you say in this case?
\end{enumerate}

{\bf{Answer:
\begin{enumerate}
    \item Assume $c^Tx_1$ is the optimal value of Boolean LP, then $0\leq x_{1i}\leq1$ since $x_{1i}\in\{0,1\}$
    \\Therefore, $x_1$ satisfies LP relaxation's constraints.
    \\Assume $c^Tx_2$ is the optimal value of LP relaxation, then $c^Tx_2\leq c^Tx_1$
    \\Hence $c^Tx_2$ is a lower bound on the optimal value of the Boolean LP 
    \\\\If LP relaxation is infeasible, then there is no x satisfying its constraints or dom.
    \\The Boolean LP's feasible solution set is contained in the LP relaxation's, which is empty.
    \\Hence Boolean LP's feasible solution set is also empty, and the problem is infeasible
    \item The LP relaxation's optimal value is also Boolean LP's, and the solution happens to be both problems' optimal solution.
\end{enumerate}
}}

\section*{Problem 3}\label{exe-sep-hyp-strict-counterexample}
Consider the QCQP
\[
\begin{array}{ll}
\mbox{minimize} & (1/2)x^T Px + q^T x + r \\
\mbox{subject to} & x^T x \leqslant 1,
\end{array}
\]
with $P\in\symm^n_{++}$.  Show that $x^\star = -(P+\lambda I)^{-1} q$
where $\lambda = \max\{0, \bar \lambda\}$
and $\bar\lambda$ is the largest solution of the nonlinear equation
\[
q^T (P+\lambda I)^{-2} q  = 1.
\]


{\bf{Answer:
\\$(1/2)x^TPx+q^Tx+r=\frac{1}{2}(x+P^{-1}q)^TP(x+P^{-1}q)+r-\frac{1}{2}q^TP^{-1}q$
\\So we just need to minimize $(x+P^{-1}q)^TP(x+P^{-1}q)=(Px+q)^TP^{-1}(Px+q)$
\\Since $P\in\symm^n_{++},(Px+q)^TP^{-1}(Px+q)\geq0$, and it's minimal when $||Px+q||_2$ is minimal 
\\Since $x$ is located in a unit circle, $||Px+q||_2$ is minimal when $Px+q=\lambda x,x=-(P+\lambda I)^{-1}q$
\\\\When $\bar{\lambda}<0,q$ is the interior point of ellipsoid $q^TP^{-2}q\leq1$
\\$x^Tx=q^TP^{-2}q\leq1$, satisfying the constraint 
\\$\lambda=0,x=-P^{-1}q,(Px+q)^TP^{-1}(Px+q)=0$
\\The optimal value is $r-\frac{1}{2}q^TP^{-1}q$
\\\\When $\bar{\lambda}\geq0,q$ is a boundary point of a larger ellipsoid $q^T(P+\bar{\lambda}I)^{-2}q\leq1$ 
\\$\forall \lambda<\bar{\lambda},q^T(P+\lambda I)^{-2}q=x^Tx>1$ because the axis is shorter
\\$\forall \lambda>\bar{\lambda},||\lambda x||_2>||\bar{\lambda} x||_2$, so $||Px+q||_2$ is minimal when $\lambda=\bar{\lambda}$
\\Therefore when $\lambda=\max\{0,\bar{\lambda}\}$, we can get the optimal value.
}}

%%%%%%%%%%%%%%%%%%%%%%%%%%%%%%%%%%%%%%%%%%%%%%%%%%%%%%%%%%%%%%%%%%%%
% Reference
%%%%%%%%%%%%%%%%%%%%%%%%%%%%%%%%%%%%%%%%%%%%%%%%%%%%%%%%%%%%%%%%%%%%
\end{document}
